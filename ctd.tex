\documentclass{article}
\usepackage{amsmath,amssymb,enumerate,bbm,calc,capt-of,ifthen}
\usepackage{graphicx}% Include figure files
\usepackage{epsfig}

\newtheorem{cntr}{do not use}
\newcommand{\tmop}[1]{\operatorname{#1}}
\newtheorem{definition}[cntr]{Definition}
\newtheorem{assumption}{Assumption}
\newcommand{\dueto}[1]{\textup{\textbf{(#1) }}}
\newtheorem{varremark}[cntr]{Remark}
\newenvironment{remark}{\begin{varremark}\em}{\em\end{varremark}}
\newcommand{\nin}{\not\in}
\newcommand{\tmem}[1]{{\em #1\/}}
\newtheorem{varnote}{Note}
\newenvironment{note}{\begin{varnote}\em}{\em\end{varnote}}
\newtheorem{proposition}[cntr]{Proposition}

\newenvironment{proof}{
  \noindent\textbf{Proof.}\ }{\hspace*{\fill}
  \begin{math}\Box\end{math}\medskip}

\newenvironment{proofof}[1]{
  \noindent\textbf{Proof of #1.}\ }{\hspace*{\fill}
  \begin{math}\Box\end{math}\medskip}

\newenvironment{proof*}[1]{
  \noindent\textbf{#1\ }}{\hspace*{\fill}
  \begin{math}\Box\end{math}\medskip}

\newtheorem{lemma}[cntr]{Lemma}
\newtheorem{corollary}[cntr]{Corollary}
\newtheorem{theorem}[cntr]{Theorem}
\newenvironment{enumeratenumeric}{\begin{enumerate}[1.]}{\end{enumerate}}
\newtheorem{algo}{Algorithm}

\numberwithin{cntr}{section}
\numberwithin{equation}{section}

\newcommand{\comment}[1]{}
%standard macros
\newcommand{\abs}[1]{\left| #1 \right|}%Absolute value
\newcommand{\absSmall}[1]{| #1 |}%Absolute value, small


%Handy macros

\newcommand{\oneto}[1]{{1 \ldots #1}}
\newcommand{\onetoN}[0]{{1 \ldots N}}
\newcommand{\Oto}[1]{{0 \ldots #1-1}}
\newcommand{\OtoN}{{0 \ldots N-1}}

\newcommand{\pointData}{{ \{ p_{i} \}_{i=\OtoN} }}
\newcommand{\tanData}{{ \{ m_{i} \}_{i=\OtoN} }}

\newcommand{\curveSet}{{ \{ \gamma_i(t) \}_{\Oto{M}}}}
\newcommand{\poly}{{\Gamma}}

\newcommand{\ball}[2]{ { B_{#1}(#2) } }

\newcommand{\kmax}{{\kappa_{m}}}
\newcommand{\kmaxi}{{\kmax^{-1}}}

\begin{document}

\title{Reconstructing Curves from Points and Tangents}

\author{L. Greengard and C. Stucchio}

\maketitle

\section{Introduction}

In this work, we consider the problem of reconstructing a $C^{1}$ family of curves $\curveSet$ from data consisting of an unorganized set of points $\pointData$, as well as \emph{unit} tangents to the points $\tanData$. Note that the tangents have no particular orientation; we identify $m_{i}$ with $-m_{i}$.

Our goal in this work is to construct an algorithm which reconstructs the curve from this data. Similar topics have been the subject of considerable attention, in particular there is much work\cite{amenta98crust,amenta98new,dey99curve,hoppe92surface,amenta02simple, dey01reconstructing} on reconstructing curves from point clouds, i.e. $\pointData$. The problem of reconstruction a curve only from point datais a rather more difficult problem; for this reason, the results available there are weaker.

//ADD ANALOGY TO INTERPOLATION//

\section{Setup}

\begin{definition}
  For a vector $v$, let $v^{\perp}$ denote the vector $v$ rotated $\pi/2$ to the right.
\end{definition}

As is normal for most interpolation theorems, we need an assumption on the curvature.

\begin{assumption}
  We assume each curve has bounded curvature:
  \begin{equation}
    \label{eq:curvatureAssumption}
    \forall i = \Oto{M}, ~ \frac{
      \abs{\gamma_{i,x}'(t) \gamma_{i,y}''(t) - \gamma_{i,y}'(t) \gamma_{i,x}''(t)}
    } {
      (\gamma_{i,x}'(t)^{2}+\gamma_{i,y}'(t)^{2})^{3/2}
    } \leq \kmax
  \end{equation}
\end{assumption}

\subsection{Basics}

We assume there is a set of curves $\curveSet$, which we call the figure. Our goal is to construct a polyganlization of $\curveSet$ from data consisting of points $\pointData$ and tangent directions at the points, $\tanData$.

\begin{definition}
  \label{def:polygonalization}
  A polygonalization graph of the set of curves $\curveSet$ is a planar graph $(V,E)$ with the following properties:
  \begin{enumerate}
  \item For all points $q \in V$, $q=\gamma_{i}(t)$ for some $i=\Oto{M}$ and some $t \in [0,1]$.
  \item The statement that $(q,q') \in E$ is equivalent to the statement that $\exists i, \exists t', t'' s.t. \gamma_{i}(t')=q, \gamma_{i}(t'')=q$ and for all $q'' \in V$, there is no $t' < \hat{t} < t''$ such that $\gamma_{i}(\hat{t}) = q''$.
  \end{enumerate}
  For the set of points $\pointData$, we let $\poly=\poly(\pointData, \curveSet)$ denote the polyonalization of $\curveSet$ with $V=\pointData$.
\end{definition}
Basically, this says that a all edges in the polygonalization graph are merely straightened versions of the edges coming from the figure. See Fig. \ref{fig:polygonalization} for an example.

\begin{figure}
\setlength{\unitlength}{0.240900pt}
\ifx\plotpoint\undefined\newsavebox{\plotpoint}\fi
\sbox{\plotpoint}{\rule[-0.200pt]{0.400pt}{0.400pt}}%
\includegraphics[scale=0.5]{polygonalization_example.eps}

\caption{A curve and it's polygonalization, c.f. Definition \ref{def:polygonalization}. }
\label{fig:polygonalization}
\end{figure}

The following basic result is
\begin{lemma}
  \label{lem:forbiddenZone}
  For every $i \neq j$, if $p_{j} \in \ball{\kmaxi}{p_{i} \pm m_{i}^{\perp} \kmaxi}$, then either:
  \begin{enumerate}[a.]
  \item The points $p_{i}$ and $p_{j}$ are not on the same curve.
  \item The the curve $\gamma_{k}(t)$ connecting $p_{i}$ to $p_{j}$ has arc length greater than $\kmaxi \pi/2$ between $p_{i}$ and $p_{j}$.
  \end{enumerate}
  We refer to the set $\ball{\kmaxi}{p_{i} \pm m_{i}^{\perp} \kmaxi}$ as the \emph{forbidden zone}; this region is illustrated in Fig. \ref{fig:forbiddenZone}.
\end{lemma}

\begin{proof}
  Suppose for simplicity that $p_{i}=(0,0)$ and $m_{i}=(1,0)$.

  Now, consider a line $\tau(t)$ of maximal curvature. The curve of maximal curvature, with $\tau_{y}'(t) > 0$ and proceeding at speed $\kmaxi$ is $\tau^{+}(t)=(\kmaxi \sin(t), \kmaxi (1-\cos(t)))$, while the curve with $\tau_{y}'(t) < 0$ is $\tau^{-}(t)=(\kmaxi \sin(t), \kmaxi (\cos(t)-1))$.

Any other curve $\gamma(t)$ must not cross one of these curves (the blue and green curves in Fig \ref{lem:forbiddenZone}). Thus, it is confined to the blue region while it's arc length is less than $\kmaxi \pi/2$. If $\gamma(t)$ connects $p_{i}$ to $p_{j}$, then it must do so after travelling a distance greater than $\kmaxi \pi/2$.
\end{proof}

\begin{figure}
\setlength{\unitlength}{0.240900pt}
\ifx\plotpoint\undefined\newsavebox{\plotpoint}\fi
\sbox{\plotpoint}{\rule[-0.200pt]{0.400pt}{0.400pt}}%
\includegraphics[scale=0.5]{forbidden_zone.eps}

\caption{The forbidden zones, as described in Lemma \ref{lem:forbiddenZone}. The pink is the forbidden zone, and the blue is the set of points a distance $\pi \kmaxi/2$ away from $p_{i}$.}
\label{fig:forbiddenZone}
\end{figure}



\bibliography{../stucchio}
\bibliographystyle{hplain}

\end{document}