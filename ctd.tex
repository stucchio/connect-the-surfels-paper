\documentclass{article}
\usepackage{amsmath,amssymb,enumerate,bbm,calc,capt-of,ifthen}
\usepackage{graphicx}% Include figure files
\usepackage{epsfig}

\newtheorem{cntr}{do not use}
\newcommand{\tmop}[1]{\operatorname{#1}}
\newtheorem{definition}[cntr]{Definition}
\newtheorem{assumption}{Assumption}
\newcommand{\dueto}[1]{\textup{\textbf{(#1) }}}
\newtheorem{varremark}[cntr]{Remark}
\newenvironment{remark}{\begin{varremark}\em}{\em\end{varremark}}
\newcommand{\nin}{\not\in}
\newcommand{\tmem}[1]{{\em #1\/}}
\newtheorem{varnote}{Note}
\newenvironment{note}{\begin{varnote}\em}{\em\end{varnote}}
\newtheorem{proposition}[cntr]{Proposition}

\newenvironment{proof}{
  \noindent\textbf{Proof.}\ }{\hspace*{\fill}
  \begin{math}\Box\end{math}\medskip}

\newenvironment{proofof}[1]{
  \noindent\textbf{Proof of #1.}\ }{\hspace*{\fill}
  \begin{math}\Box\end{math}\medskip}

\newenvironment{proof*}[1]{
  \noindent\textbf{#1\ }}{\hspace*{\fill}
  \begin{math}\Box\end{math}\medskip}

\newtheorem{lemma}[cntr]{Lemma}
\newtheorem{corollary}[cntr]{Corollary}
\newtheorem{theorem}[cntr]{Theorem}
\newenvironment{enumeratenumeric}{\begin{enumerate}[1.]}{\end{enumerate}}
\newtheorem{algo}{Algorithm}

\numberwithin{cntr}{section}
\numberwithin{equation}{section}

\newcommand{\comment}[1]{}
%standard macros
\newcommand{\abs}[1]{\left| #1 \right|}%Absolute value
\newcommand{\absSmall}[1]{| #1 |}%Absolute value, small


%Handy macros

\newcommand{\oneto}[1]{{1 \ldots #1}}
\newcommand{\onetoN}[0]{{1 \ldots N}}
\newcommand{\Oto}[1]{{0 \ldots #1-1}}
\newcommand{\OtoN}{{0 \ldots N-1}}

\newcommand{\pointData}{{ \{ p_{i} \}_{i=\OtoN} }}
\newcommand{\tanData}{{ \{ m_{i} \}_{i=\OtoN} }}

\newcommand{\curveSet}{{ \{ \gamma_i(t) \}_{\Oto{M}}}}
\newcommand{\poly}{{\Gamma}}

\newcommand{\ball}[2]{ { B_{#1}(#2) } }

\newcommand{\kmax}{{\kappa_{m}}}
\newcommand{\kmaxi}{{\kmax^{-1}}}

\begin{document}

\title{Reconstructing Curves from Points and Tangents}

\author{L. Greengard and C. Stucchio}

\maketitle

\section{Introduction}

In this work, we consider the problem of reconstructing a $C^{1}$ family of curves $\curveSet$ from data consisting of an unorganized set of points $\pointData$, as well as \emph{unit} tangents to the points $\tanData$. Note that the tangents have no particular orientation; we identify $m_{i}$ with $-m_{i}$.

Our goal in this work is to construct an algorithm which reconstructs the curve from this data. Similar topics have been the subject of considerable attention, in particular there is much work\cite{amenta98crust,amenta98new,dey99curve,hoppe92surface,amenta02simple, dey01reconstructing} on reconstructing curves from point clouds, i.e. $\pointData$. The problem of reconstruction a curve only from point datais a rather more difficult problem; for this reason, the results available there are weaker.

//ADD ANALOGY TO INTERPOLATION//

\section{Setup}

\begin{definition}
  \label{def:perp}
  For a vector $v$, let $v^{\perp}$ denote the vector $v$ rotated $\pi/2$ to the right.
\end{definition}

As is normal for most interpolation theorems, we need an assumption on the variation of the curve, in this case the curvature.

\begin{assumption}
  We assume each curve has bounded curvature:
  \begin{equation}
    \label{eq:curvatureAssumption}
    \forall i = \Oto{M}, ~ \frac{
      \abs{\gamma_{i,x}'(t) \gamma_{i,y}''(t) - \gamma_{i,y}'(t) \gamma_{i,x}''(t)}
    } {
      (\gamma_{i,x}'(t)^{2}+\gamma_{i,y}'(t)^{2})^{3/2}
    } \leq \kmax
  \end{equation}
\end{assumption}

\subsection{Basics}

We assume there is a set of curves $\curveSet$, which we call the figure. Our goal is to construct a polyganlization of $\curveSet$ from data consisting of points $\pointData$ and tangent directions at the points, $\tanData$.

\begin{definition}
  \label{def:polygonalization}
  A polygonalization of a set of curves $\curveSet$ is a planar graph $(V,E)$ with the property that each vertex $p \in V$ is a point on some $\gamma_{i}(t)$, and each edge connects points which are adjacent samples of some curve $\gamma_{i}$.

  See Fig. \ref{fig:polygonalization} for an example.
\end{definition}

\begin{figure}
\setlength{\unitlength}{0.240900pt}
\ifx\plotpoint\undefined\newsavebox{\plotpoint}\fi
\sbox{\plotpoint}{\rule[-0.200pt]{0.400pt}{0.400pt}}%
\includegraphics[scale=0.5]{polygonalization_example.eps}

\caption{A curve and it's polygonalization, c.f. Definition \ref{def:polygonalization}. }
\label{fig:polygonalization}
\end{figure}

The following basic result will be used widely in this paper, and illustrates the benefits derived from tangential information.
\begin{lemma}
  \label{lem:forbiddenZone}
  For every $i \neq j$, if $p_{j} \in \ball{\kmaxi}{p_{i} \pm m_{i}^{\perp} \kmaxi}$, then $(p_{i},p_{j})$ is not an edge in $\poly$.

  We refer to the set $\cup_{\pm} \ball{\kmaxi}{p_{i} \pm m_{i}^{\perp} \kmaxi}$ as the \emph{forbidden zone}; this region is illustrated in Fig. \ref{fig:forbiddenZone}.
\end{lemma}

\begin{proof}
  Suppose for simplicity that $p_{i}=(0,0)$ and $m_{i}=(1,0)$. Now, consider a line $\tau(t)$ of maximal curvature. The curve of maximal curvature, with $\tau_{y}'(t) > 0$ and proceeding at speed $\kmaxi$ is $\tau^{+}(t)=(\kmaxi \sin(t), \kmaxi (1-\cos(t)))$, while the curve with $\tau_{y}'(t) < 0$ is $\tau^{-}(t)=(\kmaxi \sin(t), \kmaxi (\cos(t)-1))$.

Any other curve $\gamma(t)$ must lie between these curves (the blue and green curves in Fig \ref{lem:forbiddenZone}). Thus, it is confined to the blue region while it's arc length is less than $\kmaxi \pi/2$. If $\gamma(t)$ connects $p_{i}$ to $p_{j}$, then it must do so after travelling a distance greater than $\kmaxi \pi/2$.

\begin{figure}
\setlength{\unitlength}{0.240900pt}
\ifx\plotpoint\undefined\newsavebox{\plotpoint}\fi
\sbox{\plotpoint}{\rule[-0.200pt]{0.400pt}{0.400pt}}%
\includegraphics[scale=0.5]{forbidden_zone.eps}

\caption{The forbidden zones, as described in Lemma \ref{lem:forbiddenZone}. The pink is the forbidden zone, and the blue is the set of points a distance $\pi \kmaxi/2$ away from $p_{i}$.}
\label{fig:forbiddenZone}
\end{figure}
\end{proof}

This shows that the extra information the tangents provides us can be used to exclude certain edges from the polygonalization; basically, all edges should point roughly in the direction of the tangent. This allows us to reject edges which would point in the wrong direction, yielding a more accurate polygonalization (c.f. Fig. \ref{fig:proximityVsTangentBased}).

\begin{figure}
\setlength{\unitlength}{0.240900pt}
\ifx\plotpoint\undefined\newsavebox{\plotpoint}\fi
\sbox{\plotpoint}{\rule[-0.200pt]{0.400pt}{0.400pt}}%
\includegraphics[scale=0.5]{tangents_good_for.eps}

\caption{A naive proximity-based reconstruction algorithm will add edges between separated curves which are close together. $\beta$-crust based algorithms will make similar errors, although it will have no self-intersections. However, each point on the lower arc is in the forbidden zone for points on the upper arc (indicated in pink); removing edges which enter the forbidden zones yields a correct polygonalization. }
\label{fig:proximityVsTangentBased}
\end{figure}


\bibliography{../stucchio}
\bibliographystyle{hplain}

\end{document}