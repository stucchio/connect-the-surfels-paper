\documentclass{article}
\usepackage{amsmath,amssymb,enumerate,bbm,calc,capt-of,ifthen}
\usepackage{graphicx}% Include figure files
\usepackage{epsfig}

\newtheorem{cntr}{do not use}
\newcommand{\tmop}[1]{\operatorname{#1}}
\newtheorem{definition}[cntr]{Definition}
\newtheorem{assumption}{Assumption}
\newcommand{\dueto}[1]{\textup{\textbf{(#1) }}}
\newtheorem{varremark}[cntr]{Remark}
\newenvironment{remark}{\begin{varremark}\em}{\em\end{varremark}}
\newcommand{\nin}{\not\in}
\newcommand{\tmem}[1]{{\em #1\/}}
\newtheorem{varnote}{Note}
\newenvironment{note}{\begin{varnote}\em}{\em\end{varnote}}
\newtheorem{proposition}[cntr]{Proposition}

\newenvironment{proof}{
  \noindent\textbf{Proof.}\ }{\hspace*{\fill}
  \begin{math}\Box\end{math}\medskip}

\newenvironment{proofof}[1]{
  \noindent\textbf{Proof of #1.}\ }{\hspace*{\fill}
  \begin{math}\Box\end{math}\medskip}

\newenvironment{proof*}[1]{
  \noindent\textbf{#1\ }}{\hspace*{\fill}
  \begin{math}\Box\end{math}\medskip}

\newtheorem{lemma}[cntr]{Lemma}
\newtheorem{corollary}[cntr]{Corollary}
\newtheorem{theorem}[cntr]{Theorem}
\newenvironment{enumeratenumeric}{\begin{enumerate}[1.]}{\end{enumerate}}
\newtheorem{algo}{Algorithm}

\numberwithin{cntr}{section}
\numberwithin{equation}{section}

\newcommand{\comment}[1]{}
%standard macros
\newcommand{\abs}[1]{\left| #1 \right|}%Absolute value
\newcommand{\absSmall}[1]{| #1 |}%Absolute value, small
\newcommand{\RR}[0]{{\mathbb{R}}}

%Handy macros

\newcommand{\vx}[0]{{\vec{x}}}
\newcommand{\vp}[0]{{\vec{p}}}
\newcommand{\vq}[0]{{\vec{q}}}
\newcommand{\vm}[0]{{\vec{m}}}
\newcommand{\vv}[0]{{\vec{v}}}

\newcommand{\oneto}[1]{{1 \ldots #1}}
\newcommand{\onetoN}[0]{{1 \ldots N}}
\newcommand{\Oto}[1]{{0 \ldots #1-1}}
\newcommand{\OtoN}{{0 \ldots N-1}}

\newcommand{\pointData}{{ \{ \vp_{i} \}_{i=\OtoN} }}
\newcommand{\tanData}{{ \{ \vm_{i} \}_{i=\OtoN} }}

\newcommand{\curveSet}{{ \{ \gamma_i(t) \}_{\Oto{M}}}}
\newcommand{\poly}{{\Gamma}}

\newcommand{\ball}[2]{ { B_{#1}(#2) } }
\newcommand{\allowed}[2]{ { A_{#1}(#2) } }

\newcommand{\kmax}{{\kappa_{m}}}
\newcommand{\kmaxi}{{\kmax^{-1}}}

\newcommand{\curvesep}{{\delta}}

\begin{document}

\title{Reconstructing Curves from Points and Tangents}

\author{L. Greengard and C. Stucchio}

\maketitle

\section{Introduction}

In this work, we consider the problem of reconstructing a $C^{1}$ family of curves $\curveSet$ from data consisting of an unorganized set of points $\pointData$, as well as \emph{unit} tangents to the points $\tanData$. Note that the tangents have no particular orientation; we identify $m_{i}$ with $-m_{i}$.

Our goal in this work is to construct an algorithm which reconstructs the polygonalization of the curve from this data. An example of a polygonalization is given in Figure \ref{fig:polygonalization}.

\begin{definition}
  \label{def:polygonalization}
  A polygonalization of a set of curves $\curveSet$ is a planar graph $(V,E)$ with the property that each vertex $p \in V$ is a point on some $\gamma_{i}(t)$, and each edge connects points which are adjacent samples of some curve $\gamma_{i}$.
\end{definition}

\begin{figure}
\setlength{\unitlength}{0.240900pt}
\ifx\plotpoint\undefined\newsavebox{\plotpoint}\fi
\sbox{\plotpoint}{\rule[-0.200pt]{0.400pt}{0.400pt}}%
\includegraphics[scale=0.5]{polygonalization_example.eps}

\caption{A curve and it's polygonalization, c.f. Definition \ref{def:polygonalization}. }
\label{fig:polygonalization}
\end{figure}

The topic of reconstructing curves solely from point data, i.e. $\pointData$, has been the subject of considerable attention \cite{amenta98crust,amenta98new,dey99curve,hoppe92surface,amenta02simple, dey01reconstructing}. This is actually a more difficult problem, and only weaker results are possible. The main difficulty is the following; if the distance between two separate curves $\gamma_{i}$ and $\gamma_{j}$ is smaller than the sample spacing, then it is difficult to determine which points are associated to which curve. Thus, sample spacing must be $O(\curvesep)$, with $\curvesep$ the distance between different curves.

Tangential information makes this task easier; if two points are nearby (say $\vp_{1}$ and $\vp_{2}$), but $\pm \vm_{1}$ does not point (roughly) in the direction $\vp_{2}-\vp_{1}$, then $\vp_{1}$ and $\vp_{2}$ should not be connected. This fact allows us to reduce the sample spacing to $O(\curvesep^{1/2})$, rather than $O(\curvesep)$. This is to be expected; knowledge of a function and it's derivatives allows interpolation with quadratic accuracy.

In this work, we present an algorithm which allows us to reconstruct a curve from $\pointData$ and $\tanData$. We make two assumptions, under which the algorithm is provably correct.

\begin{assumption}
  \label{ass:curvature}
  We assume each curve $\gamma_{i}$ has bounded curvature:
  \begin{equation}
    \label{eq:curvatureAssumption}
    \forall i = \Oto{M}, ~ \frac{
      \abs{\gamma_{i,x}'(t) \gamma_{i,y}''(t) - \gamma_{i,y}'(t) \gamma_{i,x}''(t)}
    } {
      (\gamma_{i,x}'(t)^{2}+\gamma_{i,y}'(t)^{2})^{3/2}
    } \leq \kmax
  \end{equation}
\end{assumption}

This assumption is necessary to prevent the curves from oscillating too much between samples.

\begin{assumption}
  \label{ass:separation}
  We assume the curves $\gamma_{i}$ and $\gamma_{j}$ are uniformly separated from each other, i.e.:
  \begin{equation}
    \label{eq:separationAssumption}
    \sup_{t,t'} \abs{ \gamma_{i}(t) - \gamma_{j}(t')} \geq \curvesep \textrm{~for~} i \neq j
  \end{equation}
  We also assume different areas of the same curve are separated from each other:
  \begin{equation}
    \label{eq:separationAssumptionSameCurve}
    \sup_{\abs{t-t'} > \kmaxi\pi/2 } \abs{ \gamma_{i}(t) - \gamma_{i}(t')} \geq \curvesep
  \end{equation}
  (assuming the curve $\gamma_{i}(t)$ proceeds with unit speed).
\end{assumption}

This assumption makes sure the curves do not come too close together for us to determine which points correspond to which curve (or which part of the same curve). This is illustrated in Figure \ref{fig:separationBetweenCurves}.

\begin{figure}
\setlength{\unitlength}{0.240900pt}
\ifx\plotpoint\undefined\newsavebox{\plotpoint}\fi
\sbox{\plotpoint}{\rule[-0.200pt]{0.400pt}{0.400pt}}%
\includegraphics[scale=0.5]{assumption_two.eps}

\caption{An illustration of Assumption \ref{ass:separation}. The black arrow illustrates \eqref{eq:separationAssumption}, while the red arrow illustrates \eqref{eq:separationAssumptionSameCurve}.}
\label{fig:separationBetweenCurves}
\end{figure}


\section{Geometry}
\subsection{Definition and notations}
Before we begin, we define some notation which we will use.

\begin{definition}
  \label{def:perp}
  For a vector $\vv$, let $\vv^{\perp}$ denote the vector $\vv$ rotated $\pi/2$ to the right.
\end{definition}

\begin{definition}
  \label{def:metric}
  Let $d(\vp,\vq)$ denote the usual Euclidean metric, $d(\vp,\vq) = \abs{\vp - \vq}$. Let $d_{\vm}(\vp,\vq)$ denote the distance in the $\vm$ direction between $\vp$ and $\vq$, i.e. $d_{\vm}(\vp,vq) = \abs{ (\vp - \vq) \cdot \vm}$.
\end{definition}

\begin{definition}
  For a point $\vp$ and a curve $\gamma_{i}(t)$, we say that $\vp \in \gamma_{i}(t)$ if $\exists t, \gamma_{i}(t)=\vp$.
\end{definition}

As is normal for most interpolation theorems, we need an assumption on the variation of the curve, in this case the curvature.

\subsection{Local results}

We assume there is a set of curves $\curveSet$, which we call the figure. Our goal is to construct a polyganlization of $\curveSet$ from data consisting of points $\pointData$ and tangent directions at the points, $\tanData$.

The following basic result will be used widely in this paper, and illustrates the benefits derived from tangential information.
\begin{lemma}
  \label{lem:forbiddenZone}
  For every $i \neq j$, if $\vp_{j} \in \ball{\kmaxi}{\vp_{i} \pm \vm_{i}^{\perp} \kmaxi}$, then $(\vp_{i},\vp_{j})$ is not an edge in $\poly$.

  We refer to the set $\cup_{\pm} \ball{\kmaxi}{\vp_{i} \pm \vm_{i}^{\perp} \kmaxi}$ as the \emph{forbidden zone}; this region is illustrated in Fig. \ref{fig:forbiddenZone}.
\end{lemma}
\begin{figure}
\setlength{\unitlength}{0.240900pt}
\ifx\plotpoint\undefined\newsavebox{\plotpoint}\fi
\sbox{\plotpoint}{\rule[-0.200pt]{0.400pt}{0.400pt}}%
\includegraphics[scale=0.5]{forbidden_zone.eps}

\caption{The forbidden zones, as described in Lemma \ref{lem:forbiddenZone}. The pink is the forbidden zone, and the blue is the set of points a distance $\pi \kmaxi/2$ away from $p_{i}$.}
\label{fig:forbiddenZone}
\end{figure}
\begin{proof}
  Suppose for simplicity that $\vp_{i}=(0,0)$ and $\vm_{i}=(1,0)$. Now, consider a line $\tau(t)$ of maximal curvature. The curve of maximal curvature, with $\tau_{y}'(t) > 0$ and proceeding at speed $\kmaxi$ is $\tau^{+}(t)=(\kmaxi \sin(t), \kmaxi (1-\cos(t)))$, while the curve with $\tau_{y}'(t) < 0$ is $\tau^{-}(t)=(\kmaxi \sin(t), \kmaxi (\cos(t)-1))$.

Any other curve $\gamma(t)$ must lie between these curves (the blue and green curves in Fig \ref{lem:forbiddenZone}). Thus, it is confined to the blue region while it's arc length is less than $\kmaxi \pi/2$. If $\gamma(t)$ connects $\vp_{i}$ to $\vp_{j}$, then it must do so after travelling a distance greater than $\kmaxi \pi/2$.

\end{proof}

This shows that the extra information the tangents provides us can be used to exclude certain edges from the polygonalization; basically, all edges should point roughly in the direction of the tangent. This allows us to reject edges which would point in the wrong direction, yielding a more accurate polygonalization (c.f. Fig. \ref{fig:proximityVsTangentBased}).

\begin{figure}
\setlength{\unitlength}{0.240900pt}
\ifx\plotpoint\undefined\newsavebox{\plotpoint}\fi
\sbox{\plotpoint}{\rule[-0.200pt]{0.400pt}{0.400pt}}%
\includegraphics[scale=0.5]{tangents_good_for.eps}

\caption{A naive proximity-based reconstruction algorithm (shown), or even $\beta$-crust type algorithms, will add edges between different curves. Knowledge of the forbidden zone (indicated in pink) allows us to remove such edges.}
\label{fig:proximityVsTangentBased}
\end{figure}

\begin{definition}
  Define the set $\allowed{\epsilon}{\vp}$ to be:
  \begin{equation}
    \label{eq:allowedRegion}
    \allowed{\epsilon}{\vp}=\ball{\epsilon}{\vp_{i}} \setminus \left[ \cup_{\pm} \ball{\kmaxi}{\vp_{i} \pm \vm_{i}^{\perp} \kmaxi} \right]
  \end{equation}
  That is, $\allowed{\epsilon}{\vp}$ is the ball of radius $\epsilon$ about $p$ excluding the forbidden zone.  We call this the \emph{allowed zone} or \emph{allowed region}.
\end{definition}
Any edge in the polygonalization starting at $\vp$, with length shorter than $\epsilon$, must connect to another point $\vp' \in \allowed{\epsilon}{p}$.

\begin{lemma}
  \label{lem:separationAllowedRegions}
  Let $i \neq j$ and let Assumption \ref{ass:separation} hold. Then for all $t, t'$, if
  \begin{subequations}
    \label{eq:separationCondition}
    \begin{equation}
      \delta > 2\kmax \epsilon^{2}
    \end{equation}
    \begin{equation}
      \label{eq:constraintOnkmaxEpsilon}
      \kmax \epsilon < \frac{1}{\sqrt{2}}
    \end{equation}
    then
    \begin{equation}
      \label{eq:connectionsBetweenDifferentCurvesNotAllowed}
      \gamma_{j}(t')  \nin  \allowed{\epsilon}{\gamma_{i}(t)}.
    \end{equation}
  \end{subequations}
  Similarly, if $i=j$ and $\abs{t-t'} \geq \kmax^{-1} \pi/2$, then \eqref{eq:connectionsBetweenDifferentCurvesNotAllowed} holds.
\end{lemma}
\begin{proof}
  Fix $t$, and define $\vp=\gamma_{i}(t)$ and $\vm=\gamma_{i}'(t) / \abs{\gamma_{i}'(t)}$. Define $L$ to be the line segment $L= \{ \vp+\vm \kmaxi \sin(\theta) : \theta \in [-\arcsin(\epsilon \kmax),\arcsin(\epsilon \kmax)] \}$. The boundaries of $\allowed{\epsilon}{\vp}$ are given by
\begin{equation*}
  \vp + \vm \kmaxi \sin(\theta) \pm \vm^{\perp} \kmaxi (1-\cos(\theta)).
\end{equation*}
Now, for any $\vq \in \gamma_{i} $ and $\vq \in \allowed{\epsilon}{\vp}$, the distance between $\vq$ and $L$ is the normal distance to $L$. This distance is bounded by:
\begin{multline}
  \label{eq:1}
  d(\vq,L) \leq
  \sup_{\theta} \kmaxi \abs{(1-\cos(\theta)) }\\
  \leq
  \sup_{\theta} \kmaxi 2 \sin^{2}(\theta/2) =
  2\kmaxi \sin^{2}( \arcsin(\epsilon \kmax)/2)
\end{multline}
The intermediate value theorem implies $\arcsin( x) \leq \arcsin'(\zeta) x=(1-\zeta^{2})^{-1/2} x$ for some $\zeta \in [0,x]$; since $\epsilon \kmax < 2^{-1/2}$ (by \eqref{eq:constraintOnkmaxEpsilon}), we find that:
\begin{equation*}
  \arcsin(\epsilon \kmax) \leq (1-(2^{-1/2})^{2})^{-1/2} \kmax \epsilon = \sqrt{2} \kmax \epsilon
\end{equation*}
Substituting this into \eqref{eq:1} yields:
\begin{equation}
  \eqref{eq:1} \leq  2 \kmaxi \sin^{2}( \sqrt{2} \kmax \epsilon/2) \leq \kmax \epsilon^{2}
\end{equation}

Thus, the \emph{normal} distance between any point in $\allowed{\epsilon}{\vp}$ and $L$ is $O(\kmax \epsilon^{2})$.

If $\gamma_{j}(t') \nin L+\vm^{\perp} \RR$, then clearly $\gamma_{j}(t') \nin \allowed{\epsilon}{\gamma_{i}(t)}$ so we assume $\gamma_{j}(t') \in L+\vm^{\perp} \RR$. In this case, $\gamma_{j}(t') = \vp + \vm \kmaxi \sin(\theta_{0}) + \vm^\perp z$, with $z \in \RR$ the normal (relative to $L$) distance to $L$.

By the second triangle inequality,
\begin{equation*}
  d_{\vm^{\perp}}(\gamma_{j}(t'), L) \geq \abs{d_{\vm^{\perp}}(\gamma_{j}, \gamma_{i}) -  d_{\vm^{\perp}}(\gamma_{i}, L)}
  \geq \delta - \kmax \epsilon^{2} > \kmax \epsilon^{2}
\end{equation*}
But this implies that $d(\gamma_{j}(t'), L) \geq d_{\vm^{\perp}}(\gamma_{j}(t'), L) \geq \kmax \epsilon^{2}$, and thus $\gamma_{j}(t') \nin \allowed{\epsilon}{\vp}$.

The proof when $i=j$ is identical.
\end{proof}

\subsection{Global results}

We now move on to global results concerning the geometry.

The following result, which follows immediately from Lemma \ref{lem:separationAllowedRegions}, shows that one can build a graph separating curves very simply.

\begin{definition}
  Fix $\epsilon$ satisfying \eqref{eq:separationCondition} and \eqref{eq:constraintOnkmaxEpsilon}.

  The allowed graph $G$ is defined to be the graph with vertex set $\pointData$, and with edges consisting of all those $(\vp_{i},\vp_{j})$, $i \neq j$ with the property that $\vp_{i} \in \allowed{\epsilon}{\vp_{j}}$ and $\vp_{j} \in \allowed{\epsilon}{\vp_{i}}$.
\end{definition}

\begin{remark}
  It is an immediate conclusion of Lemma \ref{lem:separationAllowedRegions} that $G$ has no edges connecting different curves $\gamma_{k}$ and $\gamma_{l}$. Similarly, if any points are connected in $G$, then the arc length between these points is smaller than $\kmaxi \pi/2$.
\end{remark}

\begin{proposition}
  Suppose the point data $\pointData$ is $\epsilon$-sampled, i.e. if two points $\vp_{i}$ and $\vp_{j}$ are adjacent on the curve $\gamma_{k}$, then the \emph{arc length} between $\vp_{i}$ and $\vp_{j}$ is bounded by $\epsilon$. Then $G$ contains the polygonalization of $\curveSet$.
\end{proposition}
\begin{proof}
  If the distance between adjacent points $\vp_{i}$ and $\vp_{j}$ is at most $\epsilon$, then $\vp_{j} \in \ball{\epsilon}{\vp_{i}}$. Since the segment of $\gamma_{k}$ between $\vp_{i}$ and $\vp_{j}$ has arc length less than $\epsilon$, $\vp_{j}$ is not in the forbidden zone of $\vp_{i}$ (by the same argument as in Lemma \ref{lem:forbiddenZone}. Thus, $\vp_{j} \in \allowed{\epsilon}{\vp_{i}}$ (and vice versa), and $(\vp_{i},\vp_{j})$ is an edge in $G$.
\end{proof}

We have now shown that $G$ separates distinct curves, and that $G$ contains the diagonalization of $\curveSet$.

\begin{lemma}
  Suppose the graph $G$ has
\end{lemma}

\bibliography{../stucchio}
\bibliographystyle{hplain}

\end{document}